\section{Human Simultaneous Interpretation}
\label{sec:intro}

Although simultaneous interpretation has a key role in today's international
community,\footnote{Unlike consecutive interpretation (speakers stop
  after a complete thought and wait for the interpreter), simultaneous
  interpretation has the interpreter to translate \emph{while}
  listening to speakers.} it remains underexplored within machine
translation (\abr{mt}).  One key challenge is to achieve a good
quality/speed trade-off: deciding when, what, and how to translate.
In this study, \textbf{we take a data-driven, comparative approach and
  examine:} (i) What distinguishes simultaneously interpreted text
(\inter{}\footnote{Language produced in the
  process of translation is often considered a dialect of the target
  language: ``\trans{}''~\cite{translationese}.  Thus,
  ``\inter{}'' refers to interpreted language.})
from batch-translated text (\trans{})?  (ii) What strategies do human
interpreters use?

Most previous work focuses on qualitative
analysis~\cite{epic,erik11theory,shimizu14corpus} or pattern
counting~\cite{tohyama06lrec,att13speech}.  In contrast, we use a more
systematic approach based on feature selection and statistical tests.
In addition, most work ignores \emph{translated} text, making it hard
to isolate strategies applied by interpreters as opposed to general
strategies needed for any translation.  \newcite{shimizu14corpus} are
the first to take a comparative approach; however, they directly train
\abr{mt} systems on the interpretation corpus without explicitly
examining interpretation tactics.  While some techniques can be
learned implicitly, the model may also learn undesirable behavior such
as omission and simplification: byproducts of limited human working
memory (Section~\ref{sec:analysis}).

Prior work studies simultaneous interpretation of
Japanese$\leftrightarrow$English~\cite{tohyama06lrec,shimizu14corpus}
and Spanish$\leftrightarrow$English~\cite{att13speech}.  We focus on
Japanese$\leftrightarrow$English interpretation.  Since information
required by the target English sentence often comes late in the source
Japanese sentence (e.g., the verb, the noun being modified), we expect
it to reveal a richer set of tactics.\footnote{The tactics are
  consistent with those discovered on other language pairs in prior
  work, with additional ones specific to head-final to head-initial
  languages.}  Our contributions are three-fold.  First, we collect
new human translations for an existing simultaneous interpretation
corpus, which can benefit future comparative
research.\footnote{\url{https://github.com/hhexiy/interpretese}}
Second, we use classification and feature selection methods to examine
linguistic characteristics comparatively.  Third, we categorize human
interpretation strategies, including word reordering tactics and
summarization tactics.  Our results help linguists understand
simultaneous interpretation and help computer scientists build better
automatic interpretation systems.
