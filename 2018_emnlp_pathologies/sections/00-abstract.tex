One way to interpret neural model predictions is to highlight the most
important input features---for example, a heatmap visualization over
the words in an input sentence.  In existing interpretation methods
for \nlp{}, a word's importance is determined by either input
perturbation---measuring the decrease in model confidence when that
word is removed---or by the gradient with respect to that word.  To
understand the limitations of these methods, we use input reduction,
which iteratively removes the least important word from the
input. This exposes pathological behaviors of neural models: the
remaining words appear nonsensical to humans and do not match the
words interpretation methods deem important. As we confirm with human
experiments, the reduced examples lack information to support the
prediction of any label, but models still make the same predictions
with high confidence.  To explain these counterintuitive results, we
draw connections to adversarial examples and confidence calibration:
pathological behaviors reveal difficulties in interpreting neural
models trained with maximum likelihood.  To mitigate their
deficiencies, we fine-tune the models by encouraging high entropy
outputs on reduced examples.  Fine-tuned models become more
interpretable under input reduction without accuracy loss on regular
examples.
