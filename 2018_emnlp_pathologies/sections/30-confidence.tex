\section{Making Sense of Reduced Inputs}
\label{sec:confidence}

Having established the incongruity of our definition of importance
\textit{vis-\`a-vis}
human judgements, we now investigate possible explanations for these results.
We explain why model confidence can empower methods such as \loo{} to
generate reasonable interpretations but also lead to
pathologies under input reduction. We attribute these results to two
issues of neural models.

\subsection{Model Overconfidence}
\label{sec:overconfidence}

Neural models are overconfident in their
predictions~\cite{guo2017calibration}.  One explanation for
overconfidence is overfitting: the model overfits the negative
log-likelihood loss by learning to output low-entropy distributions
over classes.  Neural models are also overconfident on examples
outside the training data distribution.  As
\citet{goodfellow2014explaining} observe for image classification,
samples from pure noise can sometimes trigger highly confident
predictions. These so-called \emph{rubbish examples} are degenerate
inputs that a human would trivially classify as not belonging to any
class but for which the model predicts with high confidence.
\citet{goodfellow2014explaining} argue that the rubbish examples exist
for the same reason that adversarial examples do: the surprising
linear nature of neural models. In short, the confidence of a neural
model is not a robust estimate of its prediction uncertainty.

Our reduced inputs satisfy the definition of rubbish examples: humans have a
hard time making predictions based on the reduced inputs
(Table~\ref{table:human_results}), but models make predictions with high
confidence (Figure~\ref{fig:confidence_density}). Starting from a valid example,
input reduction transforms it into a rubbish example.


\begin{figure}[t]
\scriptsize
\tikz\node[fill=white!90!colorsquad,inner sep=1pt,rounded corners=0.3cm]{
\begin{tabular}{p{.95\columnwidth}}
\textbf{\squad{}}\\
Context: The Panthers used the San Jose State practice facility and stayed at the San
Jose Marriott. The Broncos practiced at
\mybox{coloranswer}{Stanford University}
and stayed at the Santa Clara Marriott.\\\\
Question:\\
(0.90, 0.89) Where did the \underline{Broncos} practice for the Super Bowl ? \\
(0.92, 0.88) Where did \underline{the} practice for the Super Bowl ? \\
(0.91, 0.88) Where did practice \underline{for} the Super Bowl ? \\
(0.92, 0.89) Where did practice the Super \underline{Bowl} ? \\
(0.94, 0.90) Where did practice \underline{the} Super ? \\
(0.93, 0.90) \underline{Where} did practice Super ?\\
(0.40, 0.50) did practice Super ? \\
\end{tabular}
}; 
\caption{A reduction path for a \squad{} example. The model
    prediction is always correct and its confidence stays high (shown
    on the left in parentheses) throughout the reduction.  Each line
    shows the input at that step with an \underline{underline}
    indicating the word to remove next. The question becomes
    unanswerable immediately after ``Broncos'' is removed in the first
    step. However, in the context of the original question,
    ``Broncos'' the input gradient considers it the least important word.}
\label{fig:path}
\end{figure}


The nonsensical, almost random results are best explained by looking at a
complete reduction path (Figure~\ref{fig:path}). In this example, the transition
from valid to rubbish happens immediately after the first step: following the removal of
``Broncos'', humans can no longer determine which team the question
is asking about, but model confidence remains high.  Not being able to lower its
confidence on rubbish examples---as it is not trained to do so---the model
neglects ``Broncos'' and eventually the process generates nonsensical
results.

In this example, the \loo{} method will not highlight ``Broncos''. However, this
is not a failure of the interpretation method but of the model itself.  The model
assigns a low importance to ``Broncos'' in the first step,
causing it to be removed---\loo{} would be able to expose this
by not highlighting ``Broncos''.  However, in cases where a similar issue
only appears after a few unimportant words are removed,
\loo{} would fail to expose the unreasonable model behavior.

Input reduction can expose deeper issues of model overconfidence and
can stress test a model's uncertainty estimation and interpretability.

\begin{figure}[t]
\scriptsize
\tikz\node[fill=white!90!colorsquad,inner sep=1pt,rounded corners=0.3cm]{
\begin{tabular}{p{.95\columnwidth}}
\textbf{\squad{}} \\
Context: QuickBooks sponsored a ``Small Business Big Game'' contest, in
which Death Wish Coffee had a 30-second commercial aired free of
charge courtesy of QuickBooks. \mybox{coloranswer}{Death Wish
Coffee} beat out nine other contenders from across the United
States for the free advertisement.\\\\
Question:\\
 \mybox{color1}{\strut{What}} \mybox{color1}{\strut{company}} \mybox{color1}{\strut{won}} \mybox{color1}{\strut{free}} \mybox{color6}{\strut{advertisement}} \mybox{color1}{\strut{due}} \mybox{color1}{\strut{to}} \mybox{color1}{\strut{QuickBooks}} \mybox{color0}{\strut{\underline{contest}}} \mybox{color1}{\strut{?}}\\
 \mybox{color1}{\strut{What}} \mybox{color1}{\strut{company}} \mybox{color1}{\strut{won}} \mybox{color1}{\strut{free}} \mybox{color6}{\strut{advertisement}} \mybox{color1}{\strut{due}} \mybox{color1}{\strut{to}} \mybox{color0}{\strut{\underline{QuickBooks}}} \mybox{color1}{\strut{?}}\\
 \mybox{color2}{\strut{What}} \mybox{color0}{\strut{company}} \mybox{color1}{\strut{won}} \mybox{color1}{\strut{free}} \mybox{color0}{\strut{\underline{advertisement}}} \mybox{color1}{\strut{due}} \mybox{color1}{\strut{to}} \mybox{color5}{\strut{?}}\\
 \mybox{color1}{\strut{What}} \mybox{color0}{\strut{\underline{company}}} \mybox{color1}{\strut{won}} \mybox{color1}{\strut{free}} \mybox{color1}{\strut{due}} \mybox{color1}{\strut{to}} \mybox{color5}{\strut{?}}\\
 \mybox{color1}{\strut{What}} \mybox{color4}{\strut{won}} \mybox{color0}{\strut{\underline{free}}} \mybox{color1}{\strut{due}} \mybox{color0}{\strut{to}} \mybox{color5}{\strut{?}}\\
 \mybox{color1}{\strut{What}} \mybox{color1}{\strut{won}} \mybox{color1}{\strut{due}} \mybox{color1}{\strut{to}} \mybox{color0}{\strut{\underline{?}}}\\
 \mybox{color1}{\strut{What}} \mybox{color4}{\strut{won}} \mybox{color1}{\strut{due}} \mybox{color0}{\strut{\underline{to}}}\\
 \mybox{color4}{\strut{What}} \mybox{color2}{\strut{won}} \mybox{color1}{\strut{\underline{due}}}\\
 \mybox{color2}{\strut{What}} \mybox{color1}{\strut{\underline{won}}}\\
 \mybox{color2}{\strut{What}}\\

\end{tabular}
}; 
\caption{Heatmap generated with \loo{} shifts drastically despite only
    removing the least important word (\underline{underlined}) at each step.
    For instance, ``advertisement'', is the most important word in step two but
    becomes the least important in step three.}
\label{fig:heatmap}
\end{figure}

\subsection{Second-order Sensitivity}
\label{sec:secondorder}

Thus far, we have seen that a neural model's output changes wildly
with small changes in its input. We call this \emph{first-order}
sensitivity, because interpretation based on input gradient is a
first-order Taylor expansion of the model near the
input~\cite{simonyan2013deep}.  However, the \emph{interpretation}
(e.g., which words are important) also shifts drastically with small
input changes (Figure~\ref{fig:heatmap}). We call this
\emph{second-order} sensitivity.

The shifting heatmap suggests a mismatch between the model's first- and
second-order sensitivities. The heatmap shifts when, with respect to the removed
word, the model has low first-order sensitivity but high second-order
sensitivity.

Similar issues complicate comparable interpretation methods for image
classification models. For example, \citet{ghorbani2017interpretation}
modify image inputs so the highlighted features in the interpretation
change while maintaining the same prediction. To achieve this, they
iteratively modify the input to maximize changes in the distribution
of feature importance.  In contrast, our shifting heatmap occurs by
only removing the \emph{least} impactful features without a targeted
optimization.  They speculate that the steepest gradient direction for
the first- and second-order sensitivity values are generally
orthogonal. Loosely speaking, the shifting heatmap suggests that the
direction of the smallest gradient value can sometimes align with very
steep changes in second-order sensitivity.

When explaining individual model predictions, the heatmap
suggests that the prediction is made based on a weighted combination of words, as
in a linear model, which is not true unless the model is indeed taking a
weighted sum such as in a \abr{dan}~\cite{iyyer2015deep}. When the model
composes representations by a non-linear combination of words, a linear
interpretation oblivious to second-order sensitivity can be misleading.
