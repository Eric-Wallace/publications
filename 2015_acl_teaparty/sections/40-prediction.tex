%%%%%%%%%%%%%%%%%%%%%%%%%%%%%%%%%%%%%%%%%%%%%%%%%%%%%%%%%%%%%%%%%%%%%%%%%%%%%%%%%%%%%%%%%%%%%%%%%%%%%%%
\section{Data Collection}
\label{sec:c6_data}
%%%%%%%%%%%%%%%%%%%%%%%%%%%%%%%%%%%%%%%%%%%%%%%%%%%%%%%%%%%%%%%%%%%%%%%%%%%%%%%%%%%%%%%%%%%%%%%%%%%%%%%

What makes a Tea Partier?  To address that question, we use
\textit{key votes} identified by \fw{} as the most important votes on
issues of economic freedom.  Led by former House Majority Leader Dick
Armey (\abr{r-tx}), \fw{} is a conservative non-profit organization
which promotes ``Lower Taxes, Less Government, More
Freedom''.\footnote{\url{http://congress.freedomworks.org/}}
\newcite{Karpowitz:PS11:teaparty} report that \fw{} endorsements are
more effective than other Tea Party organizations at getting out votes
for Republican candidates in the 2010 midterms.

For the 112\textsuperscript{th} Congress, \fw{} selected 60 key votes,
40 in 2011 and 20 in 2012.  We are interested in ideal points with
respect to the Tea Party movement, i.e., on the anti-pro Tea Party
dimension: whether a legislator agrees with \fw{} on a bill. More
specifically, we assign $v \subtwo ab$ to be 1 if legislator $a$
agrees with \fw{} on bill $b$, and 0 otherwise. In addition to the votes, we obtained the bill text with labels from the Congressional Bills Project\footnote{\url{http://congressionalbills.org/}} and the congressional speeches from GovTrack.\footnote{\url{https://www.govtrack.us/data/us/112/}} In total, we have 240
Republicans, 60 who self-identify with the Tea Party Caucus, and
13,856 votes.

%%%%%%%%%%%%%%%%%%%%%%%%%%%%%%%%%%%%%%%%%%%%%%%%%%%%%%%%%%%%%%%%%%%%%%%%%%%%%%%%%%%%%%%%%%%%%%%%%%%%%%%
\section{Predicting Tea Party Membership}
\label{sec:c6_prediction}
%%%%%%%%%%%%%%%%%%%%%%%%%%%%%%%%%%%%%%%%%%%%%%%%%%%%%%%%%%%%%%%%%%%%%%%%%%%%%%%%%%%%%%%%%%%%%%%%%%%%%%%

To quantitatively evaluate the effectiveness of \name{} in
capturing ``Tea Partiness'', we predict Tea Party Caucus membership of
legislators given their votes and text. This examines (1) how effective the baseline
features extracted from the votes and text are in predicting the Caucus
membership, and (2) how much prediction improves using features extracted from \name{}.
For baselines, we consider simple feature sets where each legislator
is represented by their speeches as either (1) a \tfidf{}
vector ~\cite{salton-68}, (2) a normalized \tfidf{} vector, or (3) a binary vector
containing their votes.

Our dataset for binary prediction comprises a set of~60 Republican
representatives who self-identify as Tea Party Caucus members and~180
who do not.  These are divided using 5-fold cross-validation with stratified
sampling, which preserves the ratio of the two classes in both the training and
test sets.  We report performance using \abr{auc-roc}~\cite{lusted-71} using
\svm{}$^{light}$~\cite{Joachims:CHAP99}.\footnote{We use the default settings of
  \svm{}$^{light}$, except that we set the cost-factor equal to the ratio
  between the number of negative examples (i.e., number of non-Tea Party Caucus
  members) and the number of positive examples (i.e., number of Tea Party Caucus
  members).}  After preprocessing, our vocabulary contains
5,349 unique word types.

\jbgcomment{SVM-light's role as a baseline should be expanded,
  highlighting the connection to the model.}

\paragraph{Membership from Votes and Text.}
\label{subsec:c6_pred_vote_text}

\begin{figure}[t]
\centering
  \includegraphics[width=.9\linewidth]{\autofig{votetext_112}}
  \caption{Tea Party Caucus membership prediction results over five folds using \abr{auc-roc} (higher is better, random baseline achieves 0.5).
  The features extracted from our model are estimated using both the votes and the text.}
  \label{fig:c6_votetext_112}
\end{figure}

First, given the votes and text of all the legislators, we run \name{} for 1,000
iterations with a burn-in period of 500 iterations. After burning in, we keep
the sampled state of the model after every fifty iterations. The feature values are
obtained by averaging over the ten stored models as suggested in~\newcite{Nguyen:EMNLP14}. Each legislator $a$ is
represented by a vector concatenating:
\begin{itemize*}
  \item $K$-dimensional ideal point vector estimated from both votes and text $u \subtwo ak$
  \item $K$-dimensional vector, estimating the ideal point using only text $\bm \eta_k^T  \hat{\bm \psi} \subtwo ak$
  \item $B$ probabilities estimating $a$'s votes on $B$ bills $\Phi(x_b \sum_{k=1}^K \hat{\vartheta} \subtwo bk u \subtwo ak + y_b)$
\end{itemize*}

Figure~\ref{fig:c6_votetext_112} shows \abr{auc-roc} results for our feature
sets. \ignore{All feature sets improve upon random prediction's \abr{auc-roc}
0.5. \psrcomment{Seems unnecessary to say.}} \vote{}-based features clearly outperform text-based features like \tf{}
and \tfidf{}. Combining \vote{} with either \tf{} or \tfidf{} does not improve
the prediction performance much (i.e., \vote{}-\tf{} and
\vote{}-\tfidf{}). Features extracted from our model, \name{}, also outperform
\tf{} and \tfidf{} significantly, but only slightly better than \vote{}. However,
\name{} and \vote{} together significantly outperform \vote{} alone.

\paragraph{Membership Prediction from Text Only.}
\label{subsec:c6_pred_text}

\begin{figure}[t]
\centering
  \includegraphics[width=.85\linewidth]{\autofig{textonly_112}}
  \caption{Tea Party Caucus membership prediction results over five folds using \abr{auc-roc} (higher is better, random baseline achieves 0.5).
  The features extracted from our model for unseen legislators are estimated using their text only.}
  \label{fig:c6_textonly_112}
\end{figure}

\ignore{In the previous section, we experiment with features from both the votes and the text from
legislators to predict their Tea Party Caucus memberships. However, its applicability is limited
since we need to have both the votes and text to be able to make predictions. In this section, we
look at a more difficult, yet more practical problem, which predicts the Tea Party Caucus
membership using only the text of new lawmakers.}

The results in Figure~\ref{fig:c6_votetext_112} require both votes and
legislators' language. This is limiting, since it permits
predictions of ``Tea Partiness'' only for people with an established
congressional voting record.  A potentially more interesting and practical task
is prediction based on language alone.

Thus, we first run our inference algorithm on the training data, which includes
both votes and text.  After training, using multiple models, we sample the issue
and frame assignments for each token of the text authored by test lawmakers.
Since the votes are not available, \name{}'s extracted features here only
consist of (1) the $K$-dimensional vector $\bm \eta_k^T \hat{\bm \psi}
\subtwo ak$ estimating legislators' ideal point
using text alone, and (2) the $B$
probabilities $\Phi(x_b \sum_{k=1}^K \hat{\vartheta}
\subtwo bk u \subtwo ak + y_b)$ estimating the votes.

Figure~\ref{fig:c6_textonly_112} compares this approach with the two
baselines capable of using text alone, \tf{} and \tfidf{}.  Since \name{} can no longer access the votes in
the test data, its performance drops significantly compared with
\vote{}. However, it still quite strongly outperforms the two text-based baselines, showing that jointly modeling the voting behavior improves the text-based elements of the model.

\jbgcomment{Would be nice to have more discussion on the why}
\psrcomment{See the sentence I added at the end.}