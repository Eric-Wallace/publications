
\section{How the Tea Party Talks}
\label{sec:c6_topic}


Looking at \name{}'s induced topic hierarchy, using labeled data to
create informative priors produces highly interpretable topics at
the agenda-issue level; e.g., see the first-level nodes in
Figure~\ref{fig:frames}, which capture key issue-level debates. For example, one major
event during the 112\textsuperscript{th} Congress was the 2011 debt-ceiling
crisis, which dominates discussions in \underline{Macroeconomics}. Similarly, \underline{Defense} is
dominated by withdrawing \us{} troops from Iraq.

Turning to framing, recall that second-level nodes of the hierarchy capture
issue-specific frames of parent issues, each one associated with a frame-specific
ideal point. To analyze intra-Republican polarization, we first compute, for
each issue $k$, the \emph{span} of ideal points the frames associated with $k$,
i.e., the difference between the maximum and the minimum ideal points for frames
under that issue.\footnote{The frame proportions Dirichlet process $\psi_k$
  creates many frames with one or two observations~\cite{miller-13}.  We ignore
  those with posterior probability $\psi \subtwo kj <$ 0.1.}  We then consider several issues with a large span, i.e. whose frames are highly polarized.

\begin{figure}[t!]
\centering
  \includegraphics[width=\linewidth]{\figfile{frames}}
  \caption{Framing of \underline{Macroeconomics} (top) and \underline{Health}
    (bottom) among House Republicans, 2011-2012. Higher ideal point values are associated with the Tea
    Party.}
  \label{fig:frames}
\end{figure}

\paragraph{Macroeconomics.}

The \name{} subtree for \underline{Macroeconomics}, in Figure~\ref{fig:frames} (top), foregrounds Republican polarization related to budget issues.
 The most Tea Party oriented frame node, \underline{M3}, focuses on
criticizing government overspending, a recurring Tea Party theme.\footnote{E.g., Scott Garrett (\abr{r-nj}): ``We will not compromise on
  our principles; our principles of defending the Constitution and
  defending Americans and making sure that our posterity does not have
  this excessive debt on it.''}  In contrast, Frame~\underline{M1},
least oriented toward the Tea Party, focuses on the downsides
of a government shutdown, highlighting establishment Republican
concerns about being held responsible for the political and economic
consequences.

\paragraph{Health.}

\underline{Healthcare} was a central issue during the 112\textsuperscript{th}
Congress, particularly the Affordable Care Act (Obamacare). Although all
Republicans voted to repeal Obamacare, Figure~\ref{fig:frames} (bottom) highlights
intra-party differences in framing the issue. Frame~\underline{H1} leans strongest toward
the establishment Republican end of the spectrum, and frames opposition in terms of
the implementation of health care exchanges and the mandatory costs of the
program. In contrast, \underline{H3} captures the more strident Tea Party framing of Obamacare as an unconstitutional government takeover.

More neutral from an intra-party perspective,
Frame~\underline{H2} emphasizes Medicare, Medicaid, and the role of health
care professionals within these systems.\footnote{This does not mean that discussions using this frame
  lacked combative or partisan elements. For example, Glenn Thompson
  (\abr{r-pa}) argues that  ``on the
  Democratic side, they're just willing to pull the plug and let
  [Medicare] die''. }


\paragraph{Labor, Employment and Immigration.}

The discussion of this issue illustrates how \name{} sometimes captures frames that are distinct from Tea Partiness, \emph{per se}.  For
example, it discovered a strongly Tea Party oriented frame that focused on ``union,
south carolina, nlrb, boeing''. On inspection, this frame reflects a controversy in which the National
Labor Relations Board accused airline manufacturer Boeing of violating
Federal labor law by transferring production to a non-union facility
in South Carolina ``for discriminatory
reasons'',\footnote{http://www.nlrb.gov/news-outreach/fact-sheets/fact-sheet-archives/boeing-complaint-fact-sheet}
and surfaces mainly in speeches by four legislators from South Carolina, three of whom are from the Tea Party Caucus.


This second-level topic illustrates a limitation of \name{}; it does not formally
distinguish frames from other kinds of subtopics. We observe that modeling polarization on
other kinds of sub-issues is nonetheless valuable: here it highlights
a geographic locus of conflict involving South Carolina, where many
representatives are Tea Party Caucus members. This may provide insight
into how geography shapes Tea Party
membership~\cite{GervaisPSP12:tealeaves}.

\section{Latent and Visible Disagreement}

\begin{table}
\centering
\includegraphics[width=\linewidth]{\figfile{polarization}}
\caption{Examples of agenda issues classified by polarization of ideal points and issue frames within the Republican party.}
\label{tab:kris}
\end{table}




Our analyses suggest a novel framework for understanding the political
and policymaking implications of the Tea Party in the
112\textsuperscript{th} Congress, illustrated in
Table~\ref{tab:kris}. Each issue can be characterized by two features: (1) the degree to which ideal points among Republican legislators are polarized, and (2) the degree to which the frames used are polarized. From these two assessments, we can organize all policy issues into four categories that have meaningful implications for congressional politics and policy outcomes. At upper left we will find issues where \name{}
indicates low intra-party polarization between Tea Party and non-Tea
Party, and all Republicans tend to frame the issue in similar ways;
e.g., Civil Rights, Minority Issues, and Civil Liberties. In such
cases, we expect cooperation among Republicans regardless of Tea Party
status, therefore a greater likelihood of bill passage in a
majority-Republican House.  In stark
contrast, issues at lower right involve polarized ideal points and
polarized framing, e.g., the budget crisis, where many establishment
Republicans balked at a government shutdown but hard-line Tea Party
legislators did not. These issues pose the
greatest challenge to Republican party leaders.

Between these extremes are the issues in which \emph{either} Republicans' ideal
points \emph{or} their policy frames are polarized. Our model
suggests that on issues at upper right, with similar framing, the Tea Party and
establishment Republicans will \emph{appear} to be in sync, and therefore it may
seem to voters that legislative progress is likely, but the underlying issue
polarization will make it hard to find policy common ground, potentially
increasing public frustration. Last, at lower left are issues where Republicans
generally share similar ideal points and vote similarly, but frame the issue in
distinct ways, e.g., Obamacare. Here legislative success may come despite the appearance
that Republican factions are talking past each other, because the distribution of their ideal points on the policy is actually quite similar. Put differently, Republicans share policy goals on issues in this quadrant even if they frame those preferences differently, and this underlying agreement on the ideal point may allow Republicans to reach consensus even when the political rhetoric suggests otherwise.
