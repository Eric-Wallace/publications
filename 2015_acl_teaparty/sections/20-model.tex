
\section{Hierarchical Ideal Point Topic Model}
\label{sec:c6_model}


Bringing topic models~\cite{blei-09} into ideal-point modeling
provides an interpretable, text-based foundation for political
scientists to understand why the models make the predictions they
do. However, both the \emph{topic}---what is discussed---and the
\emph{framing}---how it is discussed---also reveal political
preferences. We therefore introduce \emph{frame-specific} ideal
points, using a hierarchy of topics to model issues and their
issue-specific frames. Although the definition of ``frame'' is itself
a moving target in political science~\cite{Entman:JC93}, we adopt the
theoretically motivated but pragmatic approach of
\newcite{Nguyen:NIPS13}: just as agenda-issues map naturally to topics
in probabilistic topic models (e.g., \newcite{Grimmer:PA10}), the
frames as second-level agenda-setting \cite{McCombs:JS05} map to
second-level topics in a hierarchical topic model.

Our model's inputs are votes $\{v \subtwo ab\}$, each the response of
legislator $a \in [1,A]$ to bill $b \in [1,B]$. Two types of text
supplement the votes: floor speeches (documents) $\{\bm w_d\}$ from
legislator $a_d$, and the text $\bm w'_b$ of bill $b$.  While
congressional debates are typically about one piece of legislation,
we make no assumptions about the mapping between $\bm w_d$ and $\bm
w'_b$. In principle this allows $\bm w_d$ to be \emph{any} text by
legislator $a_d$ (e.g., not just floor speeches about this bill, but
blogs, social media, press releases) and---unlike Gerrish and
Blei~\shortcite{Gerrish:ICML11}---this permits us to make predictions
about individuals even without vote data for
them. Figure~\ref{fig:c6_hiptm} shows the plate notation diagram of
\name{}, which has the following generative process:

\begin{small}
\begin{enumerate*}
  \item For each issue $k \in [1,K]$
  \begin{enumerate*}
    \item Draw $k$'s associated topic $\phi_k \sim~\Dir {\beta, \phi_k^{\star}}$
    \item Draw issue-specific distribution over frames $\psi_k \sim \GEM {\lambda_0}$
    \item For each frame $j \in [1, \infty)$ (specific to issue $k$)
    \begin{enumerate*}
      \item Draw $j$'s associated topic $\phi \subtwo kj \sim \Dir {\beta, \phi_k}$
      \item Draw $j$'s regression weight $\eta \subtwo kj \sim \mathcal{N}(0, \gamma)$
    \end{enumerate*}
  \end{enumerate*}
  \item For each document $d \in [1,D]$ by legislator $a_d$
  \begin{enumerate*}
    \item Draw topic (i.e., issue) distribution $\theta_d \sim~\Dir \alpha$
    \item For each issue $k \in [1,K]$, draw frame distribution $\psi \subtwo dk \sim \DP \lambda {\psi_k}$
    \item For each token $n \in [1, N_d]$
    \begin{enumerate*}
      \item Draw an issue $z \subtwo dn \sim~\Mult {\theta_d}$
      \item Draw a frame $t \subtwo dn \sim \Mult {\psi \subtwo d {z \subtwo
          dn}}$
      \item Draw word $w \subtwo dn \sim~\Mult {\phi \subtwo {z \subtwo dn}{t \subtwo
          dn}}$
    \end{enumerate*}
  \end{enumerate*}
  \item For each legislator $a \in [1, A]$ on each issue $k \in [1, K]$
  \begin{enumerate*}
    \item Draw issue-specific ideal point
    $u \subtwo ak \sim \mathcal{N}(\sum_{j=1}^{J_k} \hat{\psi} \subthree akj \eta \subtwo kj, \rho)$
      weighting $\eta \subtwo kj$ by how much the legislator talks about that frame
  \end{enumerate*}
  \item For each bill $b \in [1, B]$
  \begin{enumerate*}
    \item Draw polarity $x_b \sim \mathcal{N}(0, \sigma)$
    \item Draw popularity $y_b \sim~\mathcal{N}(0, \sigma)$
    \item Draw topic (i.e., issue) proportions $\vartheta_b \sim \Dir \alpha$
    \item For each token $m \in [1, M_b]$ in the text of bill $b$
    \begin{enumerate*}
      \item Draw an issue $z' \subtwo bm \sim~\Mult {\vartheta_b}$
      \item Draw a word type $w' \subtwo bm \sim~\Mult {\phi_{z' \subtwo bm}}$
    \end{enumerate*}
  \end{enumerate*}
  \item For each vote $v \subtwo ab$ of legislator $a$ on bill $b$
  \begin{enumerate*}
    \item $p(v \subtwo ab \given \bm u_a, x_b, y_b,\hat{\vartheta}_b) =~\Phi\left(x_b \sum_{k} \hat{\vartheta} \subtwo bk u \subtwo ak + y_b\right)$
  \end{enumerate*}
\end{enumerate*}
\end{small}

\begin{figure}[t]
\centering
  \includegraphics[width=.8\linewidth]{\figfile{hiptm}}
  \caption{Plate notation diagram of \name{}.}
  \label{fig:c6_hiptm}
\end{figure}

\paragraph{Topic Hierarchy.} With the goal of analyzing agendas and
frames in mind, our topic hierarchy has two levels: (1) \textit{issue
  nodes} and (2) \textit{frame nodes}.
(Look ahead to Figure~\ref{fig:frames} for an illustration.)
More specifically, there are
$K$ issue nodes, each with a topic $\phi_k$ drawn from a Dirichlet
distribution with concentration parameter $\beta$ and a prior mean
vector $\phi_k^{\star}$, i.e., $\phi_k \sim \Dir {\beta,
  \phi_k^{\star}}$.  In this hierarchical structure, first-level nodes
map to agenda issues, which we treat as non-polarized, and
second-level nodes map to issue-specific frames, which we assume
polarize on the issue-specific
dimension.\footnote{\newcite{Nguyen:NIPS13} allow first-level nodes to
  polarize but find first-level nodes are typically neutral.}

\begin{table}[t!]
\centering \small
\begin{tabular}{p{.95\linewidth}}
  \hline
  \textbf{{Agriculture}}:
    food; agriculture; loan; farm; crop; dairy; rural; conserve; commodity; eligible; farmer; margin; milk; contract; nutrition; livestock; plant \\ \hline
\ignore{
  \textbf{{Banking, Finance, and Domestic Commerce}}:
    insure; bank; patent; mortgage; loan; commission; issuer; director; fee; application; contract; transaction; property; internet; flood\_insurance; code; file\\ \hline
  \textbf{{Defense}}:
    unit; army\_force; transfer; army; contract; acquisition; subsection; air\_force; homeland\_security; nuclear; personnel; public\_law; nation\_defense; navy; command \\ \hline
  \textbf{{Energy}}:
    oil; electricity; fuel; shelf; outer\_continent; pipeline; facility; environment; qualification; renew\_energy; energy\_efficiency; interior; energy\_policy; exploration \\ \hline}
  \textbf{{Health}}:
    drug; medicine; coverage; disease; public\_health; hospital; social\_security; health\_insurrance; patient; application; treatment; payment; physician; nurse; clinic\\ \hline
  \textbf{{Labor, Employment, and Immigration}}:
    employment; immigration; labor; paragraph; eligible; status; compensation; application; wage; homeland\_security; unemployment; board; violation; file; perform; mine \\ \hline
\ignore{  \textbf{{Social Welfare}}:
    social\_security; disable; eligible; payment; social; food; nutrition; insurance; employment; income; poverty; earn;  calendar \\ \hline
  \textbf{{Transportation}}:
    transport; highway; motor\_vehicle; metropolitan; airport; freight; rail; carrier; chapter; driver; motor; october; traffic; paragraph; surface\_transport \\ \hline}
\end{tabular}
\caption{Examples of informed priors $\phi_k^{\star}$ for issues.}
\label{tab:c6_prior}
\end{table}

To improve topic interpretability, issue nodes have an informed prior
from the Congressional Bills Project $\{\phi_k^{\star}\}$
(Table~\ref{tab:c6_prior}).\footnote{The Congressional Bills Project
  provides a large collection of labeled congressional bill text.  We
  compute $\{\phi_k^{\star}\}$ as the empirical word distribution from
  all bills labeled with $k$. $K=19$, corresponding to 19 major topic
  headings in the Policy Agendas Project Topic Codebook. } The frame
topic $\phi \subtwo kj$ at each frame node is a Dirichlet draw
centered at the corresponding (parent) issue node. While the number of
issues is fixed \textit{a priori}, the number of second-level frames
is unbounded.  We also associate each second-level frame node with an
ideal point $\eta \subtwo kj \sim \mathcal{N}(0, \gamma)$. This
resembles how supervised topic
models~\cite{Blei:NIPS07:slda,Nguyen:NAACL15:anchor}
discover polarized topics' associated response variables.

\paragraph{Generating Text from Legislators.}
One of our model's goal is to study how legislators \textit{frame} policy agenda issues. To achieve that, we analyze congressional speeches (documents) $\{\bm w_d\}$, each of which is delivered by a legislator $a_d$. To generate each token $w \subtwo dn$ of a speech $d$, legislator $a_d$ will (1) first choose an issue $z \subtwo dn \in [1,K]$ from a document-specific multinomial $\theta_d$, then (2) choose a frame $t \subtwo dn$ from the set of infinitely many possible frames of the given issue $z \subtwo dn$ using the frame proportion $\psi \subtwo dk$ drawn from a Dirichlet process, and finally (3) choose a word type from the chosen frame's topic $\phi \subtwo {z \subtwo dn}{t \subtwo dn}$. In other words, our model generates speeches using a mixture of $K$ \hdp{}s ~\cite{Teh:JASA06:hdp}.\footnote{If we abandoned the labeled data from the Congressional Bills Project to obtain the prior means $\phi_k^{\star}$, it would be relatively straightforward to extend to a fully nonparametric model with unbounded $K$~\cite{Ahmed:ICML13:ncrf,Paisley:TPAMI14:nhdp}.}

\ignore{
One of our model's goals is to study how legislators \textit{frame}
policy agenda issues. To achieve that, we analyze congressional
speeches (documents) $\{\bm w_d\}$ from legislator $a_d$.  Our model
generates text in the speeches using a mixture of $K$
\hdp{}s~\cite{Teh:JASA06:hdp}.\footnote{If we abandoned the labeled
  data from the Congressional Bills Project to obtain the prior means
  $\phi_k^{\star}$, it would be relatively straightforward to extend
  to a fully nonparametric model with unbounded
  $K$~\cite{Ahmed:ICML13:ncrf}.}
}

\jbgcomment{There's a missing connection here.  How does it improve
  the model?}

\paragraph{Generating Bill Text.}
\label{subsec:c6_bill_text}
The bill text provides information about the policy agenda issues that each bill
addresses. We use \lda{} to model the bill text $\{\bm w'_b\}$. Each bill $b$ is
a mixture $\vartheta_b$ over $K$ issues, which is drawn from a symmetric
Dirichlet prior, i.e., $\vartheta_b \sim \Dir \alpha$. Each token $w' \subtwo
bm$ in bill $b$ is generated by first choosing a topic $z' \subtwo bm \sim \Mult
{\vartheta_b}$, and then choosing a word type $w' \subtwo bm \sim \Mult
{\phi_{z' \subtwo bm}}$, as in \lda{}.

\paragraph{Generating Roll Call Votes.}
\label{subsec:c6_vote}
Following recent work on multi-dimensional ideal
points~\cite{Lauderdale:AJPS14,Sim:AAAI15:utility}, we define the probability of
legislator $a$ voting ``Yes'' on bill $b$ as $p(v \subtwo ab = \mbox{Yes} \given
\bm u_a, x_b, y_b,\hat{\vartheta}_b) =$
\begin{equation}
\Phi\left(x_b \sum_{k=1}^K \hat{\vartheta} \subtwo bk u \subtwo ak + y_b\right)
\end{equation}
where $\hat{\vartheta}_b$ is the empirical distribution of bill $b$ over the $K$ issues and is
defined as $\hat{\vartheta} \subtwo bk = \frac{M \subtwo bk}{M \subtwo b{\cdot}}$. Here, $M \subtwo
bk$ is the number of times in which tokens in $b$ are assigned to issue $k$ and $M \subtwo
b{\cdot}$ is the marginal count, i.e., the number of tokens in bill $b$.

The ideal point of legislator $a$ specifically on issue $k$ is $u \subtwo ak$
and comes from a normal distribution
\begin{equation}
 \mathcal{N}(\hat{\psi} \subtwo ak^T \bm \eta_k, \rho) \equiv \mathcal{N}\left(\sum_{j=1}^{J_k} \hat{\psi} \subthree akj \eta \subtwo kj, \rho\right)
  \label{eqn:c6_idealpoint}
\end{equation}
where $J_k$ is the number of frames for topic $k$, which is unbounded. The mean
of the Normal distribution is a linear combination of the ideal points $\{\eta
\subtwo kj\}$ of all issue $k$'s frames, weighted by how much time legislator
$a$ spends on each frame when talking about issue $k$, i.e., $\psi \subthree akj
= \frac{N \subthree akj}{N \subthree ak{\cdot}}$. Here, $N \subthree akj$ is the
number of tokens authored by $a$ that are assigned to frame $j$ of issue $k$,
and $N \subthree ak{\cdot}$ is the marginal count. When $N \subthree ak{\cdot} =
0$, which means that legislator $a$ does not talk about issue $k$, we back off
to an uninformed zero mean.

\jbgcomment{``regress on'' should be more explicit}

Equation~\ref{eqn:c6_idealpoint} resembles how supervised topic models (\slda{}) link topics with a response, in that the response---the issue-specific ideal point $u \subtwo ak$---is latent. It is similar to how \newcite{Gerrish:ICML11} use the bill text to regress on the bill's latent polarity $x_b$ and popularity $y_b$. In this paper, we only use text from congressional speeches for regression, as these can capture how legislators frame specific topics. Incorporating the bill text into the regression is an interesting direction for future work.

\ignore{Equation~\ref{eqn:c6_idealpoint} resembles how supervised topic models (\slda{})
link topics with a response, in that the response, the issue-specific ideal
points $u \subtwo ak$, is latent. Like \newcite{Gerrish:ICML11}, we use the bill
text to regress on the bill's latent polarity $x_b$ and popularity $y_b$. We
only use text from congressional speeches for regression, as these can capture
how legislators frame specific topics. Incorporating the bill text into the
regression is an interesting direction for future work.
}









































































































































































































































































