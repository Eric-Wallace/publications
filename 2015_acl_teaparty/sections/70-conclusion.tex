
\section{Conclusion}
\label{sec:c6_conclusion}


We introduce \name{}, which integrates hierarchical topic modeling with
multi-dimensional ideal points to jointly model voting behavior, the
text content of bills, and the language used by legislators. \name{} is more effective than previous methods on
the task of predicting membership in the Tea Party Caucus. This improvement is especially consequential as the formal organization of the Tea Party Caucus is now defunct in the House, yet Tea Party legislators remain both numerous and influential in Congress. In addition, unlike previous ideal-point methods, \name{} makes it possible to make predictions for
members of Congress who have not yet established a voting record.  More
intriguingly, this also suggests the possibility of assessing the ``Tea
Partiness'' of candidates (or, anyone else, e.g., media outlets)
based on language.

It is political conventional wisdom that the influx of Tea Party legislators in
the 112\textsuperscript{th} Congress complicated the task of governance and
policymaking for Republican leaders. By looking at issue-level ideal points and
issue-specific framing using our model, we begin to address the complexity of
this relationship, finding the model successful both in establishing face
validity and in suggesting novel insights into the dynamics of a Republican
Congress.  In future work, we plan to pursue the new framework suggested by our
analyses, investigating the interaction of issue polarization
and framing-based polarization.  With the help of these new tools, we aim to
both understand and predict substantive policy areas in which the Tea Party is
likely to be most successful working with the Republican party, and, conversely,
to flag ahead of time policy areas in which we can expect to see legislative
gridlock and grandstanding.



   
