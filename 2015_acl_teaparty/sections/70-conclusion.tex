%%%%%%%%%%%%%%%%%%%%%%%%%%%%%%%%%%%%%%%%%%%%%%%%%%%%%%%%%%%%%%%%%%%%%%%%%%%%%%%%%%%%%%%%%%%%%%%%%%%%%%%
\section{Conclusion}
\label{sec:c6_conclusion}
%%%%%%%%%%%%%%%%%%%%%%%%%%%%%%%%%%%%%%%%%%%%%%%%%%%%%%%%%%%%%%%%%%%%%%%%%%%%%%%%%%%%%%%%%%%%%%%%%%%%%%%

We introduce \name{}, which integrates hierarchical topic modeling with
multi-dimensional ideal points to jointly model voting behavior, the
text content of bills, and the language used by legislators. \name{} is more effective than previous methods on
the task of predicting membership in the Tea Party Caucus. This improvement is especially consequential as the formal organization of the Tea Party Caucus is now defunct in the House, yet Tea Party legislators remain both numerous and influential in Congress. In addition, unlike previous ideal-point methods, \name{} makes it possible to make predictions for
members of Congress who have not yet established a voting record.  More
intriguingly, this also suggests the possibility of assessing the ``Tea
Partiness'' of candidates (or, anyone else, e.g., media outlets)
based on language.

It is political conventional wisdom that the influx of Tea Party legislators in
the 112\textsuperscript{th} Congress complicated the task of governance and
policymaking for Republican leaders. By looking at issue-level ideal points and
issue-specific framing using our model, we begin to address the complexity of
this relationship, finding the model successful both in establishing face
validity and in suggesting novel insights into the dynamics of a Republican
Congress.  In future work, we plan to pursue the new framework suggested by our
analyses, investigating the interaction of issue polarization
and framing-based polarization.  With the help of these new tools, we aim to
both understand and predict substantive policy areas in which the Tea Party is
likely to be most successful working with the Republican party, and, conversely,
to flag ahead of time policy areas in which we can expect to see legislative
gridlock and grandstanding.

\ignore{
\emph{Possible text to include in intro/conclusion}

As has been noted earlier, the Tea Party can be hard to define. Political pundits, journalists and scholars remain divided in classifying the movement as a party faction or a social movement as well as whether its rise is attributed to grassroots or elite support. As a result, when studying the Tea Party in the U.S. House of Representatives, it is most common to use legislators’ self-identification with the Tea Party Caucus, a voluntary member organization within Congress (see Gervais and Morris 2014).  But as this paper reveals, some legislators have high levels of “tea partiness” but are not members of the Tea Party Caucus, while other legislators join the Tea Party Caucus but their voting behavior reveals a more moderate set of positions. Being able to use text to identify legislators who have a high level of tea party-ness on different issues is important for understanding legislative policymaking.  Given the Tea Party’s divisive role within the Republican party, understanding where the dividing lines lie helps us to understand party power, leadership contests, and of course, policy outcomes.


Furthermore it may become harder in the future to rely exclusively on membership in the Tea Party Caucus to identify Tea Party legislators there is no guarantee that the Tea Party Caucus will continue to exist, especially since its’ founder, Michele Bachmann (R-MN), is no longer in the House.  Another challenge is that the Tea Party Caucus is not as well established in the U.S. Senate due to the less extensive caucus system in that chamber. Even at the start of the 112th Congress in 2011, only two months after the success of the Tea Party in the 2010 elections, the Tea Party Caucus in the Senate attracted only 4 members and did not attract senators who were widely seen as Tea Party darlings during the 2010 election such as Marco Rubio (R-FL) and Pat Toomey (R-PA). Therefore, the approach used here not only provides a more accurate assessment of which legislators’ behavior is most consistent with the Tea Party, but it also provides a valuable tool that is not dependent on the Tea Party Caucus.
}

\ignore{Its
  hierarchical structure follows \newcite{Nguyen:NIPS13} in
  providing a well defined computational realization of agenda setting and
  framing, motivated by political science theory~\cite{McCombs:JS05}. Moreover,
  its multi-dimensionality permits modeling of legislators as having different
  ideological leanings on different issues, and its leveraging of labeled data
  from the Congressional Bills Project enhances interpretability to aid analysis by political domain experts.}  \ignore{Looking at
  Republican votes and debates in the 112\textsuperscript{th} \us{} House of
  Representatives, analysis with single-dimensional ideal point modeling shows
  that Tea Party and non-Tea Party Republicans overlap significantly in
  ideological preference, which is unsurprising and not particularly
  informative. Moving to \name{}, though, we discover a richer picture of the
  relationship between Tea Party and ``establishment'' Republicans, one that
  takes advantage of the hierarchy's first level to illuminate the focus of
  Congressional floor debates, and the hierarchy's second level to illuminate
  differences in framing by the two sometimes agreeing, sometimes conflicting
  groups of legislators.} 
