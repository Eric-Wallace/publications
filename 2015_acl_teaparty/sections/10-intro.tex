
\section{Capturing Political Polarization}
\label{sec:c6_intro}


Ideal-point models are one of the most widely used tools in
contemporary political science research \cite{Poole:Book07:ideology}.
These models estimate political preferences for legislators, known as
their \emph{ideal points}, from binary data such as legislative votes.
Popular formulations analyze legislators' votes and place them on a
one-dimensional scale, most often interpreted as an ideological
spectrum from liberal to conservative.

Moving beyond a single dimension is attractive, however, since people
may lean differently based on policy issues; for example, the
conservative movement in the \us{} includes fiscal conservatives
who are relatively liberal on social issues, and vice versa. In
\textit{multi-dimensional} ideal point models, therefore, the ideal
point of each legislator is no longer characterized by a single
number, but by a multi-dimensional
vector. With that move comes a new challenge, though: the additional
dimensions are often difficult to interpret. To mitigate this problem,
recent research has introduced methods that estimate multi-dimensional
ideal points using both voting data and the texts of the bills being
voted on, e.g., using topic models and associating each dimension of
the ideal point space with a topic. The words most strongly associated with the topic can sometimes
provide a readable description of its corresponding dimension.

In this paper, we develop this idea further by introducing \name{},
the \textit{Hierarchical Ideal Point Topic Model}, to estimate
multi-dimensional ideal points for legislators in the \us{}
Congress. \name{} differs from previous models in three ways. First,
\name{} uses not only votes and associated bill text, but also the
\emph{language} of the legislators themselves; this allows predictions
of ideal points from politicians' writing alone. Second, \name{}
improves the interpretability of ideal-point dimensions by
incorporating data from the Congressional Bills Project~\cite{CBP}, in
which bills are labeled with major topics from the Policy
Agendas Project Topic Codebook.\footnote{\url{http://www.policyagendas.org/}} And third,
\name{} discovers a \emph{hierarchy} of topics, allowing us to analyze
both agenda issues and issue-specific frames that legislators use on
the congressional floor, following \newcite{Nguyen:NIPS13} in modeling
framing as second-level agenda setting~\cite{McCombs:JS05}.

Using this new model, we focus on Republican legislators during
the~112\textsuperscript{th} \us{} Congress, from January 2011 until
January 2013.  This is a particularly interesting session of Congress
for political scientists, because of the rise of the Tea Party, a
decentralized political movement with populist, libertarian, and
conservative elements.  Although united with ``establishment''
Republicans against Democrats in the 2010~midterm elections,
leading to massive Democratic defeats, the Tea Party was---and still
is---wrestling with establishment Republicans for control of the
Republican party.

The Tea Party is a new and complex phenomenon for political
scientists; as \newcite{CarminesARPS15} observe: ``Conventional views
of ideology as a single-dimensional, left-right spectrum experience
great difficulty in understanding or explaining the Tea Party.''  Our
model identifies legislators who have low (or high) levels of ``Tea
Partiness'' but are (or are not) members of the Tea Party Caucus, 
and providing insights into the nature of polarization within the
Republican party. \name{} also makes it possible to investigate a number of questions of interest to
political scientists. For example, are there Republicans who identify
themselves as members of the Tea Party, but whose votes and language
betray a lack of enthusiasm for Tea Party issues?  How well can we
predict from someone's language alone whether they are likely to
associate themselves with the Tea Party?  Our computational modeling approach to ``Tea
Partiness'', distinct from self-declared Tea Party Caucus membership, 
may have particular value in understanding Republican
party politics going forward because, despite the continued
influence of the Tea Party, the official Tea Party Caucus in the
House of Representatives is largely inactive and its future uncertain~\cite{fuller2015}.








\section{Polarization across Dimensions}
\label{subsec:c6_idealpoint_overview}


Ideal point models describe probabilistic
relationships between observed responses (votes) on a set of items (bills) by a set of
responders (legislators) who are characterized by continuous latent
traits~\cite{Fox:Book10}.  A popular formulation posits an
\textit{ideal point} $u_a$ for each lawmaker $a$, a \textit{polarity}
$x_b$, and \textit{popularity} $y_b$ for each bill $b$, all being
values in
$(-\infty,+\infty)$~\cite{Martin:PA02,Bafumi:PA05,Gerrish:ICML11}. Lawmaker
$a$ votes ``Yes'' on bill $b$ with probability
\begin{equation}
  p(v \subtwo ab = \mbox{Yes} \given u_a, x_b, y_b) = \Phi(u_a x_b + y_b)
  \label{eqn:c6_onedim}
\end{equation}
where $\Phi(\alpha) = \exp(\alpha) / (1 + \exp(\alpha))$ is the
logistic (or inverse-logit) function.\footnote{A probit function is
  also often used where $\Phi(\alpha)$ is instead the cumulative
  distribution function of a Gaussian
  distribution~\cite{Martin:PA02}.} Intuitively, most lawmakers
vote ``Yes'' on bills with high popularity $y_b$ and ``No'' on bills
with low $y_b$. When a bill's popularity is lower, the outcome of
the vote $v \subtwo ab$ depends more on the interaction between the lawmaker's ideal
point $u_a$ and the bill's polarity $x_b$.

Multi-dimensional ideal point models replace scalars $u_a$ and $x_b$
with $K$-dimensional vectors $\bm u_a$ and $\bm
x_b$~\cite{Heckman:NBER96,Jackman:PA01,Clinton:APSR04}.
Unfortunately, as \newcite{Lauderdale:AJPS14} observe, the binary data
used for these models are ``insufficiently informative to support
analyses beyond one or two dimensions'', and the additional dimensions
are difficult to interpret.  To address this lack of interpretability, recent work has proposed multi-dimensional ideal point models to jointly capture both binary votes and the associated text~\cite{Gerrish:NIPS12,Gu:KDD14,Lauderdale:AJPS14,Sim:AAAI15:utility}.


