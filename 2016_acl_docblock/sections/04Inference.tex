\section{Posterior Inference}
\label{sec:inference}

Posterior inference (Algorithm~\ref{alg:sampling}) consists of the sampling of
topic and block assignments and the optimization of weight vectors and
matrix.\footnote{More details about sampling procedures and equations in this
  section (including the sampling and optimization equations using sigmoid loss)
  are available in the supplementary material.}  We add an auxiliary variable
$\bm{\lambda}$ for hinge loss (see Section~\ref{ssec:topic_sampling}), so the
updating of~$\bm{\lambda}$ is not necessary when using sigmoid loss.

\begin{algorithm}[t!]
\caption{Sampling Process}\label{alg:sampling}
\begin{algorithmic}[1]
    \State Set $\bm{\lambda}=1$ and initialize topic assignments \label{alg-line:init}
    \For{$m=1$ to $M$}
        \State Optimize $\bm{\eta}$, $\bm{\tau}$, and $\bm{\rho}$ using \lbfgs \label{alg-line:opt}
        \For{$d=1$ to $D$}
            \State Draw block assignment $y_d$ \label{alg-line:block}
            \For{each token $n$}
                \State Draw a topic assignment $z_{d,n}$ \label{alg-line:topic}
            \EndFor
            \For{each explicit link $(d, d')$}
                \State Draw $\lambda_{d,d'}^{-1}$ (and then $\lambda_{d,d'}$) \label{alg-line:lambda}
            \EndFor
        \EndFor
    \EndFor
\end{algorithmic}
\end{algorithm}

The sampling procedure is an iterative process after initialization
(Line~\ref{alg-line:init}).  In each iteration, we first optimize the
weight vectors and matrix (Line~\ref{alg-line:opt}) before updating
documents' block assignments (Line~\ref{alg-line:block}) and topic
assignments (Line~\ref{alg-line:topic}).  When using hinge loss, the
auxiliary variable~$\bm{\lambda}$ for every explicit link needs to be
updated (Line~\ref{alg-line:lambda}).

\subsection{Sampling Block Assignments}
\label{ssec:block_sampling}

Block assignment sampling is done by Gibbs sampling, using the block
assignments and links in~$\bm{A}$ excluding document~$d$ and its
related links.\footnote{These equations deal with undirected edges,
  but they can be adapted for directed edges. See supplementary
  material.}  The probability that~$d$ is assigned to block~$l$
is\par\nobreak
\begin{small}
\begin{align}
\Pr ( &y_d=l \,|\, \bm{A}_{-d}, \bm{y}_{-d}, a, b, \gamma ) \propto \bigbrack{N_l^{-d}+\gamma} \times \notag\\
&\prod_{l'} \frac {\bigbrack{S_e^{-d}(l,l') + b}^{S_w^{-d}(l,l') + a}} {\bigbrack{S_e^{-d}(l,l') + b + S_e(d,l')}^{S_w^{-d}(l,l') + a + S_w(d,l')}} \notag\\
&\prod_{i=0}^{S_w(d,l') - 1} \bigbrack{S_w^{-d}(l,l') + a + i}, \label{equ:block_sampling}
\end{align}
\end{small}
\noindent where~$N_l$ is the number of documents assigned to block~$l$;
$^{-d}$ denotes that the count excludes document~$d$;
$S_w(d,l)$ and~$S_w(l,l')$ are the sums of link weights from document~$d$ to block~$l$ and from block~$l$ to block~$l'$, respectively:\par\nobreak
\begin{small}
\begin{align}
S_w(d,l) =& \sum_{d': y_{d'} = l} A_{d,d'}\\
S_w(l,l') =& \sum_{d: y_d = l} S_w(d,l').
\end{align}
\end{small}

\noindent $S_e(d,l)$ is the number of possible links from document~$d$
to~$l$ (i.e., assuming document~$d$ connects to every document in
block~$l$), which equals~$N_l$.  The number of possible links from
block~$l$ to~$l'$ is~$S_e(l,l')$ (i.e., assuming every document in
block~$l$ connects to every document in block~$l'$):
\begin{equation}
\small
S_e(l,l')=
\left\{
    \begin{array}{cc}
        N_l\times N_{l'} & l\neq l'\\
        \frac{1}{2} N_l (N_l-1) & l=l'.
    \end{array}
\right.
\end{equation}

If we rearrange the terms of Equation~\ref{equ:block_sampling} and put
the terms which have~$S_w(d,l')$ together, the value of~$S_w(d,l')$
increases (i.e., document~$d$ is more densely connected with documents
in block~$l'$), the probability of assigning~$d$ to block~$l$
decreases exponentially. Thus if~$d$ is more densely connected with
block~$l$ and sparsely connected with other blocks, it is
more likely to be assigned to block~$l$.

\subsection{Sampling Topic Assignments}
\label{ssec:topic_sampling}

































Following~\newcite{polson-2011-data}, by introducing an auxiliary
variable~$\lambda_{d,d'}$, the conditional probability of assigning~$t_{d,n}$, the $n$-th token in document~$d$, to topic~$k$
is\par\nobreak
\begin{small}
\begin{align}
\Pr (z_{d,n}&=k \,|\,\bm{z}_{-d,n}, \bm{w}_{-d,n}, w_{d,n}=v, y_d=l) \notag\\
\propto&\bigbrack{N_{d,k}^{-d,n} + \alpha \pi_{l,k}^{-d,n}} \frac {N_{k,v}^{-d,n} + \beta} {N_{k,\cdot}^{-d,n} + V\beta} \notag\\
&\prod_{d'} \exp \bigbrack{- \frac {(\zeta_{d,d'}+\lambda_{d,d'})^2} {2\lambda_{d,d'}}}, \label{equ:topic_sampling}
\end{align}
\end{small}
where~$N_{d,k}$ is the number of tokens in document~$d$ that are
assigned to topic~$k$; $N_{k,v}$ denotes the count of word~$v$
assigned to topic~$k$; Marginal counts are denoted by~$\cdot$;
$^{-d,n}$ denotes that the count excludes~$t_{d,n}$; $d'$ denotes all
documents that have explicit links with document~$d$.  The block topic
prior~$\pi_{l,k}^{-d,n}$ is estimated based on the maximal path
assumption~\cite{cowans-2006-thesis,wallach-2008-thesis}:
\begin{equation}\label{equ:pi}
\small
\pi_{l,k}^{-d,n} = \frac {\sum_{d': y_{d'}=l} N_{d',k}^{-d,n} + \alpha'} {\sum_{d': y_{d'}=l} N_{d',\cdot}^{-d,n} + K\alpha'}.
\end{equation}
the link prediction argument $\zeta_{d,d'}$ is
\begin{equation}\label{equ:zeta}
\small
\zeta_{d,d'} = 1 - B_{d,d'} \bigbrack{\frac{\eta_k}{N_{d,\cdot}} \frac{N_{d',k}}{N_{d',\cdot}} + R_{d,d'}^{-d,n}}.
\end{equation}
where\par\nobreak
\begin{small}
\begin{align}
R_{d,d'}^{-d,n} =& \sum_{k=1}^{K} \eta_{k} \frac{N_{d,k}^{-d,n}}{N_{d,\cdot}} \frac{N_{d',k}}{N_{d',\cdot}} + \notag\\
&\sum_{v=1}^{V} \tau_{v} \frac{N_{d,v}}{N_{d,\cdot}} \frac{N_{d',v}}{N_{d',\cdot}} + \rho_{y_d, y_{d'}} \Omega_{y_d, y_{d'}}.
\end{align}
\end{small}

Looking at the first term of Equation~\ref{equ:topic_sampling}, the
probability of assigning~$t_{d,n}$ to topic~$k$ depends not only on
its own topic distribution, but also the topic distribution of the
block it belongs to.  The links also matter: Equation~\ref{equ:zeta}
gives us the intuition that a topic which could increase the
likelihood of links is more likely to be selected, which forms an
interaction between topics and the link graph---the links are guiding
the topic sampling while updating topic assignments is maximizing the
likelihood of the link graph.

\subsection{Parameter Optimization}
\label{ssec:opt}

While topic assignments are updated iteratively, the weight vectors
and matrix~$\bm{\eta}$,~$\bm{\tau}$, and~$\bm{\rho}$ are optimized in
each global iteration over the whole corpus using
\lbfgs~\cite{liu-1989-lbfgs}. It takes the likelihood of generating~$\bm{B}$ using~$\bm{\eta}$,~$\bm{\tau}$,~$\bm{\rho}$, and current topic
and block assignments as the objective function, and optimizes it
using the partial derivatives with respect to every weight
vector/matrix element.















The log likelihood of~$\bm{B}$ using hinge loss is\par\nobreak
\begin{small}
\begin{align}
\mathcal{L} (\bm{B}) \propto& -\sum_{d,d'} \frac{R_{d,d'}^2 - 2(1+\lambda_{d,d'})B_{d,d'}R_{d,d'}}{2\lambda_{d,d'}} \notag\\
&- \sum_{k=1}^{K} \frac{\eta_k^2}{2\nu^2} - \sum_{v=1}^{V} \frac{\tau_v^2}{2\nu^2} - \sum_{l=1}^{L} \sum_{l'=1}^{L} \frac{\rho_{l,l'}^2}{2\nu^2}.
\end{align}
\end{small}

We also need to update the auxiliary variable~$\lambda_{d,d'}$.  Since
the likelihood of~$\lambda_{d,d'}$ follows a generalized inverse
Gaussian distribution $\mathrm{GIG} \bigbrack{\lambda_{d,d'};
  \frac{1}{2}, 1, \zeta_{d,d'}^2}$, we sample its reciprocal~$\lambda_{d,d'}^{-1}$ from an inverse Gaussian distribution as
\begin{equation}\label{equ:lambda}
\small
\prob{\lambda_{d,d'}^{-1}}{\bm{z}, \bm{w}, \bm{\eta}, \bm{\tau}, \bm{\rho}} = \mathrm{IG} \bigbrack{\lambda_{d,d'}^{-1}; \frac{1}{|\zeta_{d,d'}|}, 1}.
\end{equation}
