\section{\dknn{} is Interpretable}
\label{sec:interpretation}

Following past work~\cite{li2016understanding, murdoch2018cd}, we focus on the
\abr{sst} dataset for generating interpretations. Due to the lack of standard
interpretation evaluation metrics~\cite{doshivelez2017towards},
we use qualitative evaluations~\cite{smilkov2017smoothgrad,
sundararajan2017axiomatic, li2016understanding}, performing
quantitative experiments where possible to examine the distinction
between the interpretation methods.

\subsection{Interpretation Analysis}


\begin{table*}[t]
\centering
\begin{tabular}{lp{0.6\textwidth}}
\toprule
\textbf{Method} & \textbf{Saliency Map} \\
\midrule
Conformity &  \mybox{color5}{an} \mybox{color1}{intelligent} \mybox{color5}{fiction} \mybox{color5}{about} \mybox{color5}{learning} \mybox{color5}{through} \mybox{color5}{cultural} \mybox{color5}{clash}. \\
Confidence &  \mybox{color5}{an} \mybox{color1}{intelligent} \mybox{color6}{fiction} \mybox{color5}{about} \mybox{color5}{learning} \mybox{color5}{through} \mybox{color5}{cultural} \mybox{color9}{clash}. \\
Gradient &  \mybox{color5}{an} \mybox{color2}{intelligent} \mybox{color8}{fiction} \mybox{color5}{about} \mybox{color5}{learning} \mybox{color5}{through} \mybox{color5}{cultural} \mybox{color8}{clash}. \\
\midrule
Conformity &  \mybox{color5}{$<$Schweiger$>$} \mybox{color5}{is} \mybox{color2}{talented} \mybox{color5}{and} \mybox{color5}{terribly} \mybox{color2}{charismatic}. \\
Confidence &  \mybox{color6}{$<$Schweiger$>$} \mybox{color5}{is} \mybox{color2}{talented} \mybox{color5}{and} \mybox{color8}{terribly} \mybox{color2}{charismatic}. \\
Gradient &  \mybox{color6}{$<$Schweiger$>$} \mybox{color5}{is} \mybox{color2}{talented} \mybox{color5}{and} \mybox{color9}{terribly} \mybox{color2}{charismatic}. \\
\midrule
Conformity &  \mybox{color5}{Diane} \mybox{color5}{Lane} \mybox{color2}{shines} \mybox{color5}{in} \mybox{color5}{unfaithful}. \\
Confidence &  \mybox{color2}{Diane} \mybox{color5}{Lane} \mybox{color2}{shines} \mybox{color5}{in} \mybox{color6}{unfaithful}. \\
Gradient &  \mybox{color4}{Diane} \mybox{color5}{Lane} \mybox{color2}{shines} \mybox{color5}{in} \mybox{color8}{unfaithful}. \\
\bottomrule

\end{tabular}
\vspace{8pt}
\begin{tabular}{c}
Color Legend\quad  \mybox{color2}{Positive Impact}\quad \mybox{color7}{Negative Impact}
\end{tabular}
\caption{Comparison of interpretation approaches on \abr{sst} test
examples for the LSTM model. Blue indicates positive impact and red
indicates negative impact. Our method (\textit{Conformity}
\loo{}) has higher precision, rarely assigning
importance to extraneous words such as ``about'' or ``movie''.}
\label{table:saliency_maps}
\end{table*}


We compare our method (\emph{Conformity \loo{}}) against two baselines: \loo{}
using regular confidence (\emph{Confidence \loo{}}, see Section \ref{sec:l10}) and the gradient with respect to the
input (\emph{Gradient}, see Section \ref{sec:gradient}). To create saliency maps, we normalize each word's importance
by dividing it by the total importance of the words in
the sentence. We display unknown words in angle brackets $<>$.
Table~\ref{table:saliency_maps} shows \abr{sst} interpretation examples
for the \bilstm{} model and further examples are shown on a
supplementary website.\footnote{\url{https://sites.google.com/view/language-dknn/}}

Conformity \loo{} assigns concentrated importance values to
a small number of input words. In contrast, the baseline methods
assign non-zero importance values to numerous words, many of which
are irrelevant.
For instance, in all three examples of Table~\ref{table:saliency_maps}, 
both baselines highlight almost half of the input, including words such as
``fiction'' and ``clash''. We suspect model confidence is oversensitive to
these unimportant input changes,
causing the baseline interpretations to highlight unimportant words. On the other hand, 
the conformity score better separates word importance, generating clearer interpretations.

The tendency for confidence-based approaches to assign importance 
to many words holds for the entire test set. We compute the average number of
highlighted words using a threshold of $0.05$ 
(a normalized importance value corresponding to a \mybox{color4}{light blue} or \mybox{color6}{light red} highlight). Out
of the average 20.23 words in \abr{sst} test set, gradient highlights 5.32 words,
confidence \loo{} highlights 5.79 words, and conformity \loo{} highlights 3.65 words.

The second, and related, observation for confidence-based approaches is a bias
towards selecting word importance based on a word's inherent sentiment,
rather than its meaning in context. For example, see ``clash'',
``terribly'', and ``unfaithful'' in Table \ref{table:saliency_maps}.
The removal of these words causes a small change in the model confidence.
When using \dknn{}, the conformity score indicates that the model's
uncertainty has not risen without these input words and \loo{} does not
assign them any importance.

We characterize our interpretation method as significantly
higher precision, but slightly lower recall than confidence-based
methods. Conformity \loo{} rarely assigns high importance to words
that do not align with
human perception of sentiment. However, there are cases when our
method does not assign significant importance to any word. This occurs
when the input has a high redundancy. For example, a positive movie
review that describes the sentiment in four distinct ways. In these
cases, leaving out a single sentiment word has little effect on the
conformity as the model's representation remains supported by the
other redundant features. Confidence-based interpretations, which
interpret models using the linear units that produce class scores,
achieve higher recall by responding to every change in the input for a
certain direction but may have lower precision.

In the second example of Table~\ref{table:saliency_maps}, the word ``terribly''
is assigned a negative importance value, disregarding its positive meaning in context.
To examine if this is a stand-alone example or a more general pattern of 
uninterpretable behavior, we calculate the importance value of the word
``terribly'' in other positive examples. For each occurrence of the word ``great''
in positive validation examples, we paraphrase it to ``awesome'', ``wonderful'',
or ``impressive'', and add the word ``terribly'' in front of it. 
This process yields $66$ examples. For each of these examples,
we compute the importance value of each input word and rank them from
most negative to most positive (the most negative word has
a rank of 1). We compare the average ranking of ``terribly''
from the three methods: 7.9 from conformity \loo{}, 1.68 from confidence \loo{}, and 1.1 from
gradient. The baseline methods consistently rank ``terribly'' as the most
negative word, ignoring its meaning in context. This echoes our suspicion: \dknn{} generates
interpretations with higher precision because conformity is robust to irrelevant input changes.



\subsection{Analyzing Dataset Annotation Artifacts}
\label{sec:degenerate}

We use conformity \loo{} to interpret a model trained on \abr{snli}, a dataset known to
contain annotation artifacts. We demonstrate that our interpretation
method can help identify when models exploit dataset biases.

Recent studies~\cite{gururangan2018annotation,poliak2018hypothesis}
identify annotation artifacts in \abr{snli}. Superficial
patterns exist in the input which strongly correlate with certain
labels, making it possible for models to ``game'' the task: obtain high
accuracy without true understanding. For instance, the hypothesis of an
entailment example is often a general paraphrase of the premise, using
words such as ``outside'' instead of ``playing in a park''. Contradiction
examples often contain negation words or non-action verbs like ``sleeping''.
Models trained solely on the hypothesis can learn these patterns and reach
accuracies considerably higher than the majority baseline.

These studies indicate that the \abr{snli} task can be gamed. We look to confirm
that some artifacts
are indeed exploited by normally trained models that use full input pairs. We
create saliency maps for examples in the validation set using conformity
\loo{}. Table~\ref{table:annotation_artifacts} shows samples and more can be found on the
supplementary website. We use
blue highlights to indicate words which positively support the model's predicted class,
and the color red to indicate words that support a different class.
The first example is a randomly sampled baseline, showing how the words
``swims'' and ``pool'' support the model's prediction of contradiction.
The other examples are selected because they contain terms identified as artifacts.
In the second example, conformity \loo{} assigns extremely high word importance
to ``sleeping'', disregarding the other words necessary to predict contradiction (i.e., the neutral class is still possible
if ``pets'' is replaced with ``people''). In the final two hypotheses, the interpretation
method diagnoses the model failure, assigning high importance to ``wearing'', rather than
focusing positively on the shirt color.

To explore this further, we analyze the hypotheses in each \abr{snli} class
which contain a top five artifact identified by \citet{gururangan2018annotation}.
For each of these examples,
we compute the importance value for each input word using
both confidence and conformity \loo{}. We then rank the words from
most important for the prediction to least important (a score of 1
indicates highest importance) and report the average rank for the artifacts in Table~\ref{table:words}.
We sort the words by their Pointwise Mutual Information with the correct label as provided by \citet{gururangan2018annotation}.
The word ``nobody'' particularly stands out: it is the most important input word every time it appears
in a contradiction example. 

For most of the artifacts, conformity \loo{} assigns them a high importance,
often ranking the artifacts as the most important input word. 
Confidence \loo{} correlates less strongly with the known artifacts, frequently ranking them
as low as the fifth or sixth most important word. Given the high correlation between 
conformity \loo{} and the manually identified artifacts, this interpretation method may serve as
a technique to identify undesirable biases a model has learned.


\begin{table*}[t]
\centering
\begin{tabular}{llp{0.6\textwidth}}
\toprule
\textbf{Prediction} & \textbf{Input} & \textbf{Saliency Map} \\
\midrule
\multirow{2}{*}{Contradiction} & Premise & a young boy reaches for and touches the propeller of a vintage aircraft.  \\
& Hypothesis &  \mybox{color5}{a} \mybox{color8}{young} \mybox{color6}{boy} \mybox{color2}{swims} \mybox{color4}{in} \mybox{color6}{his} \mybox{color4}{pool}. \\
\midrule
\multirow{2}{*}{Contradiction} & Premise & a brown dog and a black dog in the edge of the ocean with a wave under them boats are on the water in the background.  \\
& Hypothesis &  \mybox{color5}{the} \mybox{color5}{pets} \mybox{color5}{are} \mybox{color2}{sleeping} \mybox{color5}{on} \mybox{color5}{the} \mybox{color5}{grass}. \\
\midrule
& Premise & man in a blue shirt standing in front of a structure painted with geometric designs.\\
Entailment & Hypothesis &  \mybox{color5}{a} \mybox{color6}{man} \mybox{color5}{is} \mybox{color2}{wearing} \mybox{color5}{a} \mybox{color6}{blue} \mybox{color5}{shirt}.\\
\textcolor{red}{Entailment} & Hypothesis &  \mybox{color5}{a} \mybox{color6}{man} \mybox{color5}{is} \mybox{color2}{wearing} \mybox{color5}{a} \mybox{color6}{black} \mybox{color5}{shirt}.\\
\bottomrule

\end{tabular}
\vspace{8pt}
\begin{tabular}{c}
Color Legend\quad  \mybox{color2}{Positive Impact}\quad \mybox{color7}{Negative Impact}
\end{tabular}
\caption{Interpretations generated with conformity \loo{} align with
annotation biases identified in \abr{snli}. In the second example, the
model puts emphasis on the word ``sleeping'', disregarding other words
that could indicate the Neutral class. The final example diagnoses a
model's incorrect Entailment prediction (shown in red). Green
highlights indicate words that support the classification decision
made (shown in parenthesis), pink highlights indicate words that
support a different class.}
\label{table:annotation_artifacts}
\end{table*}


\begin{table}
\setlength{\tabcolsep}{2pt}
\centering
\small
\begin{tabular}{clcc}
\toprule
\textbf{Label} & \textbf{Artifact} & \textbf{Conformity} & \textbf{Confidence} \\
\midrule
\multirow{5}{*}{\textbf{Entailment}}
& outdoors & 2.93 & 3.26 \\
& least & 2.22 & 4.41 \\
& instrument & 3.57 & 4.47 \\
& outside & 4.08 & 4.80 \\
& animal & 2.00 & 4.73 \\
\midrule
\multirow{5}{*}{\textbf{Neutral}}
& tall & 1.09 & 2.61 \\
& first & 2.14 & 2.99 \\
& competition & 2.33 & 5.56 \\
& sad & 1.39 & 1.79 \\
& favorite & 1.69 & 3.89 \\
\midrule
\multirow{5}{*}{\textbf{Contradiction}}
& nobody & 1.00 & 1.00 \\
& sleeping & 1.64 & 2.34 \\
& no & 2.53 & 5.74 \\
& tv & 1.92 & 3.74 \\
& cat & 1.42 & 3.62 \\
\bottomrule
\end{tabular}
\caption{The top \abr{snli} artifacts identified by \citet{gururangan2018annotation} are shown on the left. For each word, we compute the average importance rank over the validation set using either \textit{Conformity} or \textit{Confidence} \loo{}. A score of 1 indicates that a word is always ranked as the most important word in the input. \textit{Conformity} \loo{} assigns higher importance to artifacts, suggesting it better diagnoses model biases.}
\label{table:words}
\end{table}
