









We introduce \name{}, an interactive framework that combines active
learning \emph{selection}s with topic model \emph{overview}s to help
users induce a label set and label documents. We show that users can
more effectively and efficiently induce label set and create training
data using \name{} in comparison with other conditions, which lack
either topic \emph{overview} or active \emph{selection}.

We can further improve \name{} (the \abr{ta} condition) to help users gain
better and faster understanding of text corpora.  Our current system limits
users to view only $20K$ documents at a time and allows for one label assignment
per document. Moreover, the topics are static and do not adapt to better reflect
users' labels. Users should have better support for browsing documents and
assigning multiple labels.  Topics can also improve
via \abr{slda}~\cite{blei-07b} or
\abr{llda}~\cite{ramage2009labeled} as users add labels.

Finally, with slight changes to what the system
considers a document, we believe \name{} can be extended to \abr{nlp} applications
other than classification, such as named entity recognition or
semantic role labeling, to reduce the annotation effort.
