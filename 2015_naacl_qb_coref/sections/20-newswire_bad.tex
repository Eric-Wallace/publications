\section{Newswire's Limitations for Coreference}
\label{sec:newswire-bad}

\begin{table*}
\begin{tabular*}{\linewidth}{p{0.05\linewidth}|p{0.9\linewidth}@{}}
\hline
\textbf{NW} & Later, [they]$_1$ all met with [President Jacques Chirac]$_2$. [Mr. Chirac]$_2$ said an important first step had been taken to calm tensions. \\
\textbf{NW} & Around the time of the [Macau]$_1$ handover, questions that were hot in [the Western media]$_2$ were ``what is Macaense''? And what is native [Macau]$_1$ culture?  \\
\textbf{NW} & [MCA]$_1$ said that [it]$_1$ expects [the proposed transaction]$_2$ to be completed no later than November 10th. \\
\hline
\textbf{QB} & As a child, [this character]$_1$ reads [[his]$_1$ uncle]$_2$ [the column]$_3$ [\emph{That Body of Yours}]$_3$ every Sunday. \\
\textbf{QB} & At one point, [these characters]$_1$ climb into barrels aboard a ship bound for England. Later, [one of [these characters]$_1$]$_2$ stabs [the Player]$_3$ with a fake knife. \\
\textbf{QB} & [One poet from [this country]$_2$]$_1$ invented the haiku, while [another]$_3$ wrote the [\emph{Tale of Genji}]$_4$. Identify [this homeland]$_2$ of [Basho]$_1$ and [Lady Murasaki]$_3$. \\
\hline
\end{tabular*}
\caption{Three newswire sentences and three quiz bowl sentences with annotated coreferences and singleton mentions. These examples show that quiz bowl sentences contain more complicated types of coreferences that may even require world knowledge to resolve.}
\label{table1}
\end{table*}

Newswire text is widely used as training data for coreference
resolution systems. The standard datasets used in the
\abr{muc}~\cite{MUC-6,MUC-7},
\abr{ace}~\cite{doddington2004automatic}, and \conll{} shared
tasks~\cite{Pradhan:2011:CST:2132936.2132937} contain only such
text. In this section we argue why this monoculture, despite its many
past successes, offer diminishing results for advancing the
coreference subfield.












First, newswire text has sparse references, and those that it has are
mainly identity coreferences and appositives. In the \conll{} 2011
shared task~\cite{pradhan2007ontonotes} based on OntoNotes
4.0~\cite{hovy2006ontonotes},\footnote{As our representative for
  ``newswire'' data, the English portion of the Ontonotes 4.0 contains
  professionally-delivered weblogs and newsgroups (15\%), newswire
  (46\%), broadcast news (15\%), and broadcast conversation (15\%).}
there are 2.1 mentions per sentence; in the next section we present a
dataset with 3.7 mentions per sentence.\footnote{Neither of these
  figures include singleton mentions, as OntoNotes does not have gold
  tagged singletons. Our dataset has an even higher density when
  singletons are included.} In newswire text, most nominal entities
(not including pronouns) are singletons; in other words, they do not
corefer to anything. OntoNotes 4.0 development data contains 25.4K
singleton nominal entities~\cite{DurrettKlein2013}, compared to only
7.6K entities which corefer to something (anaphora). On the other
hand, most pronominals are anaphoric, which makes them easy to resolve
as pronouns are single token entities. While it is easy to obtain a
lot of newswire data, the amount of coreferent-heavy mention clusters
in such text is not correspondingly high.

Second, coreference resolution in news text is trivial for humans because it
rarely requires world knowledge or semantic understanding. Systems trained on
news media data for a related problem---entity extraction---falter on
non-journalistic texts~\cite{poibeau2001proper}.  This discrepancy in
performance can be attributed to the stylistic conventions of
journalism. Journalists are instructed to limit the number of entities mentioned
in a sentence, and there are strict rules for referring to
individuals~\cite{boyd-08}. Furthermore, writers cannot assume that their
readers are familiar with all participants in the story, which requires that
each entity is explicitly introduced in the text~\cite{goldstein2004associated}.
These constraints make for easy reading and, as a side effect, easy coreference
resolution. Unlike this simplified ``journalistic'' coreference, everyday
coreference relies heavily on inferring the identities of people and entities in
language, which requires substantial world knowledge.

While news media contains examples of coreference, the primary goal of a
journalist is to convey information, not to challenge the reader's coreference
resolution faculty. Our goal is to evaluate coreference systems on data that
taxes even human coreference.





