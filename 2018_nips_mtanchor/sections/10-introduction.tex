\section{Introduction: Exploring multilingual document collections}
\label{sec:intro}


Modeling multilingual topics aids exploration of large corpora across
languages~\citep{mimno-2009}.  These models help align topics
cross-lingually and uncover latent relationships between languages,
such as observing the differences in describing economic issues
between English and Spanish speakers~\citep{gutierrez-2016}.
Incorporating multilingual information also forms better monolingual
topics~\citep{jagarlamudi-2010}.




Multilingual topic models usually depend on some resource to bridge languages.  These resources include word
alignments~\citep{zhao-2006}, dictionaries~\citep{jagarlamudi-2010,
  boyd-graber-2010}, topic alignments in documents~\citep{ni-09}, or
all of the above~\citep{hu-2014-ptlda}.  Existing
multilingual models have several shortcomings; they assume
  extensive knowledge about languages, preclude human
  refinement, and are slow. Thus, a topic model may not be
appropriate in emergent sitations on low resource languages when time
is of the essence: e.g., when relief workers must triage relief
messages in Hatian Creole~\citep{morrow-2011}.


Beyond these practical concerns, adding interactivity to topic
modeling allows machine learning non-experts to build models better
suited to their needs~\citep{choo-2013, hu-2014-itm, lee-2017}.  One way to quickly
incorporate human knowledge into the model is through 
anchor words~\citep{lund-2017}.  Inference in anchor-based topic
models is driven by anchors, which are words that have high
probability in one topic and low probability in remaining
topics~\citep{arora-2012-anchor, arora-2013}.  The anchoring algorithm scales with the number of unique word types, making it fast enough for interactive updates.

We present two contributions for modeling multilingual topics.  First,
we develop a multilingual anchoring algorithm, which is an extension to
anchor-based topic inference for comparable corpora.\footnote{Comparable corpora across
  languages are collections of documents about the same themes but
  that are \emph{not} translations.  Compared to more typical parallel
  data~\citep{mauro-2012,graff-1994}, comparable data are more
  challenging.}  Second, we introduce \mtanchor, a human-in-the-loop
system that uses multilingual anchoring to align topics and enables
users to make further adjustments to the model.\footnote{\url{http://github.com/forest-snow/mtanchor_demo}.} Through interaction, the model produces
\emph{interpretable}, low-dimensional representations of documents.
These vector representations improve intra-lingual or
cross-lingual text classification. The topic model generates
coherent topic aligments for comparable corpora
because users themselves align topics.








